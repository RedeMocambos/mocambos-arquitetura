\section{Django}\label{django}
Django é um framework, escrito em Python, para o desenvolvimento
rápido de aplicações web, também conhecido como ``\emph{The web
  framework for perfectionists with deadlines}''\footnote{``O
  framework web para perfeccionistas com prazos''.}, pois foi criado
com o objetivo de respeitar o máximo possível os princípios
DRY\footnote{``Don't Repeat Yourself, não se repita, (DRY, também
  conhecido como ``Single Point of Truth'', ponto único de verdade) é
  um principio segundo o qual a informação não tem que ser repetida e
  redundante e não tem que expressar o mesmo conceito mais de uma
  vez.'', traduzido do Wikipedia:
  \url{http://it.wikipedia.org/wiki/Don\%27t_Repeat_Yourself}.}, e
então oferece todas as ferramentas para escrever código limpo de
maneira pragmática e eficaz.

As caraterísticas do Django são:
\begin{itemize}
\item é um framework \emph{Model, View, Template}, (MVT),
  correspondente ao difundido \emph{pattern Model, View, Controller},
  (MVC). O \emph{Template} do Django corresponde ao \emph{view} de
  qualquer framework MVC, enquanto a \emph{View} corresponde ao
  \emph{controller} (mesmo se as funcionalidades do \emph{controller},
  no caso do Django, não são limitadas a componente \emph{View}).
\item permite modelar os dados diretamente em python e o
  \emph{Object Relational Mapper} (ORM), se ocupa de transformá-los em
  código SQL. Também não precisa escrever \emph{query} diretamente em
  SQL; é só usar o ORM para obter os resultados diretamente sob forma
  de objetos python. O ORM de Django suporta PostgreSQL, MySQL, sqlite,
  Microsoft SQL Server e Oracle.
\item possui uma biblioteca para \emph{form} realmente poderosa e
  expressiva que permite realizar logicas complexas com poucas linhas
  de código, deixando o máximo controle ao desenvolvedor para todas as
  situações.
\item é um framework ``\emph{batteries included}'', ou seja já vêm com
  algumas aplicações, chamadas \emph{apps}, para as funcionalidade
  mais comuns, que reduzem o tempo de desenvolvimento, por exemplo um
  sistema de autenticação, uma interface de administração, um
  framework para gerar \emph{sitemaps XML} e muito mais. Além disso a
  comunidade é bem ativa e existem muitas \emph{apps} adaptáveis e
  reusáveis.
\item é entre os melhores framework quanto a documentação
  oficial. Além de alguns tutoriais, uteis para aprender as bases,
  todas as funcionalidades e características do framework são bem
  documentadas.
\end{itemize}

\section{Django-CMS}\label{django-cms}
Django CMS adiciona uma serie de funcionalidades já prontas ao
framework Django. Possibilita criar paginas em blocos. Em cada bloco
podem se colocar conteúdos através dos plugins. Por padrão existem já
varios plugins para gerir imagens, arquivos, vídeos, etc\ldots. Alem
disso existem muitos plugins desenvolvidos pela comunidade e criar
novos é relativamente simples.

\section{Zinnia}\label{zinnia}
O Zinnia é uma aplicação para criar blog no django e se integra bem ao
django-cms, providenciando já alguna plugins para inserir nas páginas
do django-cms conteúdos do blog zinnia.  Um artigo/post no blog é
representado pelo modelo/classe ZinniaEntry que tem a seguinte
estrutura\footnote{É possível visualizar a estrutura das
  classes/modelos do django com o comando $python manage.py
  inspectdb$}:

\begin{code}
class ZinniaEntry(models.Model):
    id = models.IntegerField(primary_key=True)
    title = models.CharField(max_length=255)
    image = models.CharField(max_length=100)
    content = models.TextField()
    excerpt = models.TextField()
    tags = models.CharField(max_length=255)
    slug = models.CharField(max_length=255)
    status = models.IntegerField()
    featured = models.BooleanField()
    comment_enabled = models.BooleanField()
    pingback_enabled = models.BooleanField()
    creation_date = models.DateTimeField()
    last_update = models.DateTimeField()
    start_publication = models.DateTimeField(null=True, blank=True)
    end_publication = models.DateTimeField(null=True, blank=True)
    login_required = models.BooleanField()
    password = models.CharField(max_length=50)
    template = models.CharField(max_length=250)
    class Meta:
        db_table = u'zinnia_entry'
\end{code}


\subsection{Importando as notícias do velho portal}
O Zinnia inclui um script para importar conteúdos a partir de um
RSS. O script usa a ótima biblioteca python
\emph{feedparser}\footnote{\url{http://packages.python.org/feedparser/}}.
Vamos alterar esse script para conseguir armazenar as imagens das
noticias do portal Plone no novo portal. Precisa achar um jeito de
analizar o conteúdo das notícias, coletar todos os links as imagens
para em seguida salva-las no novo portal e atualizar os links. É
possível extender o \emph{ZinniaEntry} para incluir um campo imagem
(na real uma lista de imagens). Por fim alterar também o template do
ZinniaEntry para visualizar as imagens associadas ao objeto, por
exemplo:

\begin{code}
        
	<div id="content">
        
             <img src="{{ image }}" alt="{{ Entry.title }}"/>
        
        </div>
        
\end{code}

Para extender o Entry precisa criar um modelo abstrato que herde do
EntryBaseModel. Desta forma é possivel adicionar campos/atributos e sobrescrever as
funções/metodos[\ref{ExtZinniaEntry}]. 

Nesse momento precisa so criar mais um atributo para armazenar uma
lista de imagens. Poderia ser uma lista de objetos imagens ou
simplesmente uma lista de link. No caso vamos criar so uma lista de
links/url.

Voltando ao script de importação \emph{feed2zinnia}. Vamos criar uma
função para procurar os links das imagens dentro da descrição/conteúdo
do feed:

\begin{code}
def import_images(self, feed_entry):
    soup = BeautifulSoup(feed_entry.description)
    return [x['src'] for x in soup.findAll('img', {'class': 'image-inline'})]
\end{code}

Agora precisa pegar esses links as imagens e inserir no novo atributo
Entry.images_list[].

Vamos alterar então a função \emph{import_entries} dessa maneira:

\begin{code}
categories = self.import_categories(feed_entry)
            images_links = self.import_images(feed_entry)
            entry_dict = {'title': feed_entry.title[:255],
                          'content': feed_entry.description,
                          'excerpt': feed_entry.get('summary'),
                          'status': PUBLISHED,
                          'creation_date': creation_date,
                          'start_publication': creation_date,
                          'last_update': timezone.now(),
                          'slug': slug,
                          'images_links': images_links}
\end{code}

\section{Alterar o artigo para incluir as imagens}
O Zinnia tem uma boa documentação para extender a classe que define o
artigo\ref{ExtZinniaEntry}. Precisa criar uma nova aplicação para
django, ou seja criar uma pasta e criar um arquivo vazio $$
__init__.py $$. Vamos criar um objeto para representar as imagens que
precisa anexar aos artigos. Para isso criamos um arquivo $$
models.py $$:

\begin{code}
from django.db import models
from zinnia.models import EntryAbstractClass, Entry

class EntryMocambosImage(models.Model):
    image = models.ImageField(upload_to='zinnia_images')
    entry = models.ForeignKey(Entry, related_name='%(class)s_images')
\end{code}

Depois criamos o nosso novo modelo que nesse momento nao contem quase
nada, chamado, por exemplo, modelos.py, desta forma:

\begin{code}
from django.db import models
from zinnia.models import EntryAbstractClass

class EntryMocambos(EntryAbstractClass):

    class Meta:
    abstract = True
\end{code}

Precisa registrar as alterações no sistema de administração:

\begin{code}
from django.contrib import admin
from zinnia.models import Entry
from zinnia.admin.entry import EntryAdmin
from mocamboszinnia.models import EntryMocambosImage
from django.utils.translation import ugettext_lazy as _
from django.db.models.loading import cache as model_cache
if not model_cache.loaded:
    model_cache.get_models()


class EntryMocambosImageInline(admin.TabularInline):
    model = EntryMocambosImage
    extra = 3

class EntryMocambosAdmin(EntryAdmin):
    inlines = [ EntryMocambosImageInline, ]
    # In our case we put the gallery field                                                                                                                     
    # into the 'Content' fieldset                                                                                                                              
    fieldsets = ((_('Content'), {'fields': (
                    'title', 'content', 'image', 'status')}), ) + \
                    EntryAdmin.fieldsets[1:]


admin.site.unregister(Entry)
admin.site.register(Entry, EntryMocambosAdmin)
\end{code}


\chapter{Anexos}
\section{Extendendo o modelo do artigo do blog do Zinnia}\label{ExtZinniaEntry}
% Vince_TODO: Traduzir
In fact, simply by creating an abstract model inherited from
EntryBaseModel, adding fields or/and overriding his methods, and
registering it with the ZINNIA_ENTRY_BASE_MODEL setting in your
project.

Example for adding a gallery field.

\begin{code}
from django.db import models
from mygalleryapp.models import Gallery
from zinnia.models import EntryAbstractClass

class EntryGallery(EntryAbstractClass):
  gallery = models.ForeignKey(Gallery)

  class Meta:
    abstract = True
\end{code}

Now you register the EntryGallery model like this in your project’s settings.

$$ ZINNIA_ENTRY_BASE_MODEL = 'appname.custom_entry.EntryGallery' $$

Finally extend the entry’s admin class to show your custom field.

\begin{code}
from django.contrib import admin
from zinnia.models import Entry
from zinnia.admin.entry import EntryAdmin
from django.utils.translation import ugettext_lazy as _

class EntryGalleryAdmin(EntryAdmin):

  # In our case we put the gallery field
  # into the 'Content' fieldset
  fieldsets = ((_('Content'), {'fields': (
    'title', 'content', 'image', 'status', 'gallery')})) + \
    EntryAdmin.fieldsets[1:]

admin.site.unregister(Entry)
admin.site.register(Entry, EntryGalleryAdmin)
\end{code}

You can see another example in the files zinnia/plugins/placeholder.py and zinnia/plugins/admin.py.
