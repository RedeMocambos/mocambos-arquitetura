Estou procurando um jeito de armazenar as imagens das noticias do
portal Plone no novo portal. Precisa achar um jeito de coletar todos
os links e salvar as imagens localmente. Talvez posso extender o
ZinniaEntry para incluir um campo imagem. Assim consigo relacionar o
link ao objeto e criar a partir de template, com algo do tipo:

<code>
	<div id="content">
	     <img src="{Entry.image}" alt="{Entry.title}"/>
	     </div>
</code>

Para extender o Entry precisa criar um modelo abstrato que herde do
EntryBaseModel, adicionando os campos e sobrescrevendo as
funções/metodos[\ref{ExtZinniaEntry}]. 

Voltando ao script de importação \emph{feed2zinnia}, por enquanto ficou:

<code>
	for feed_entry in feed_entries:
	    soup = BeautifulSoup(feed_entry.description)
            #print feed_entry.description                                                                                                                  
            #result = soup.find("img")["src"]
	    # AQUI PRECISA PEGAR OS LINKS DAS IMAGENS E COLAR NOS OBJETOS. Criar um atributo de tipo lista no entry.                                                                                                               
            print([x['src'] for x in soup.findAll('img', {'class': 'image-inline'})])
                #if result is not None:                                                                                                                                       #    print result                                                                                                                                             #self.write_out(self.style.NOTICE(soup.find("img")["src"]))               
</code>

Agora precisa pegar os links das images e colocar numa lista associada
ao objeto. Para isso criamos o atributo Entry.images[].

O codigo poderia ser entao:
<code>

	Entry.images = [x['src'] for x in soup.findAll('img', {'class': 'image-inline'})]

</code>




############################################################################

\chapter{ExtZinniaEntry}\label{ExtZinniaEntry}
In fact, simply by creating an abstract model inherited from
EntryBaseModel, adding fields or/and overriding his methods, and
registering it with the ZINNIA_ENTRY_BASE_MODEL setting in your
project.

Example for adding a gallery field.

from django.db import models
from mygalleryapp.models import Gallery
from zinnia.models import EntryAbstractClass

class EntryGallery(EntryAbstractClass):
  gallery = models.ForeignKey(Gallery)

  class Meta:
    abstract = True

Now you register the EntryGallery model like this in your project’s settings.

ZINNIA_ENTRY_BASE_MODEL = 'appname.custom_entry.EntryGallery'

Finally extend the entry’s admin class to show your custom field.

from django.contrib import admin
from zinnia.models import Entry
from zinnia.admin.entry import EntryAdmin
from django.utils.translation import ugettext_lazy as _

class EntryGalleryAdmin(EntryAdmin):

  # In our case we put the gallery field
  # into the 'Content' fieldset
  fieldsets = ((_('Content'), {'fields': (
    'title', 'content', 'image', 'status', 'gallery')})) + \
    EntryAdmin.fieldsets[1:]

admin.site.unregister(Entry)
admin.site.register(Entry, EntryGalleryAdmin)

You can see another example in the files zinnia/plugins/placeholder.py and zinnia/plugins/admin.py.
